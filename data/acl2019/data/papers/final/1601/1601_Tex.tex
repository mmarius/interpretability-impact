%
% File acl2018.tex
%
%% Based on the style files for ACL-2017, with some changes, which were, in turn,
%% Based on the style files for ACL-2015, with some improvements
%%  taken from the NAACL-2016 style
%% Based on the style files for ACL-2014, which were, in turn,
%% based on ACL-2013, ACL-2012, ACL-2011, ACL-2010, ACL-IJCNLP-2009,
%% EACL-2009, IJCNLP-2008...
%% Based on the style files for EACL 2006 by 
%%e.agirre@ehu.es or Sergi.Balari@uab.es
%% and that of ACL 08 by Joakim Nivre and Noah Smith

\documentclass[11pt,a4paper]{article}
%\usepackage{hyperref}
\usepackage{acl2018}
\usepackage{times}
\usepackage{latexsym}

%\usepackage[backend=biber]{biblatex}
\usepackage{url}

\usepackage{booktabs}
\usepackage{multirow}

%\usepackage[whole]{bxcjkjatype} % いっぱい日本語したい(あとで消す)
\usepackage{microtype} % suppress hyphenations
\usepackage{outlines} % easy itemization
\usepackage{amsmath,amssymb}
\usepackage{mathtools}
\mathtoolsset{showonlyrefs=true} % equation number only if it's referred
\usepackage{bm} % bold math
\usepackage{caption}
\usepackage{subcaption}
\usepackage{xspace}
\usepackage{makecell} % line breaks inside a table cell

\makeatletter
\newcommand\footnoteref[1]{\protected@xdef\@thefnmark{\ref{#1}}\@footnotemark}
\makeatother

\aclfinalcopy % Uncomment this line for the final submission
\def\aclpaperid{1601} %  Enter the acl Paper ID here

%\setlength\titlebox{5cm}
% You can expand the titlebox if you need extra space
% to show all the authors. Please do not make the titlebox
% smaller than 5cm (the original size); we will check this
% in the camera-ready version and ask you to change it back.

\newcommand\BibTeX{B{\sc ib}\TeX}

\newcommand\todo[1]{\textcolor{red}{#1}}
\newcommand{\mat}[1]{\bm{#1}}
\renewcommand{\vec}[1]{\bm{#1}}
\newcommand{\setoftriples}{\mathcal{T}}
\newcommand{\setofents}{\mathcal{E}}
\newcommand{\setofrels}{\mathcal{R}}
\newcommand{\kb}{\mathcal{G}}
\newcommand{\loss}{\mathcal{L}}
\newcommand{\ent}[1]{\textit{#1}}
\newcommand{\rel}[1]{\textit{#1}}
\newcommand{\lexrel}[2]{\rel{#1}_{(\ent{#2})}}
\newcommand{\triple}[3]{(\ent{#1}, \rel{#2}, \ent{#3})}
\renewcommand{\path}{\pi}
\newcommand{\onetoone}{\textsc{1-to-1}\xspace}
\newcommand{\oneton}{\textsc{1-to-N}\xspace}
\newcommand{\ntoone}{\textsc{N-to-1}\xspace}
\newcommand{\nton}{\textsc{N-to-N}\xspace}
\newcommand{\mtriple}[3]{\langle #1, #2, #3 \rangle}

\DeclareMathOperator*{\relu}{ReLU}
\DeclareMathOperator*{\tr}{tr}

\title{Interpretable and Compositional Relation Learning by Joint Training with an Autoencoder}

\author{Ryo Takahashi*\textsuperscript{1} \and
  Ran Tian*\textsuperscript{1} \and
  Kentaro Inui\textsuperscript{1,2}\\
  {\bf (* equal contribution)}\\
  \textsuperscript{1}Tohoku University\quad\textsuperscript{2} RIKEN, 
  \quad Japan \\
  {\tt \{ryo.t,\;tianran,\;inui\}@ecei.tohoku.ac.jp}
  \\}

\date{}

\begin{document}
\maketitle
\begin{abstract}
Embedding models for entities and relations are extremely 
useful for recovering missing facts in a knowledge base. Intuitively, a relation can be modeled by 
a matrix mapping entity vectors. However, relations reside 
on low dimension sub-manifolds in the parameter space of arbitrary matrices -- 
for one reason, composition of two relations $\mat{M}_1,\mat{M}_2$ may match a third $\mat{M}_3$
(e.g. composition of relations \texttt{currency\_of\_country} and \texttt{country\_of\_film} usually matches 
\texttt{currency\_of\_film\_budget}), which imposes compositional 
constraints to be satisfied by the parameters (i.e. $\mat{M}_1\cdot \mat{M}_2\approx \mat{M}_3$). 
In this paper we investigate a dimension reduction technique by training relations 
jointly with an autoencoder, which is expected to 
better capture compositional 
constraints. We achieve state-of-the-art on Knowledge Base Completion 
tasks with strongly improved Mean Rank, and show that joint training with 
an autoencoder leads to interpretable sparse codings of relations, helps discovering 
compositional constraints and benefits from compositional training. Our source code is released at
\url{github.com/tianran/glimvec}.
\end{abstract}

\section{Introduction}

Broad-coverage knowledge bases (KBs) such as %YAGO~\citep{DBLP:conf/www/SuchanekKW07} and 
Freebase~\citep{DBLP:conf/sigmod/BollackerEPST08} and 
DBPedia~\citep{auer2007dbpedia}
store a large amount of facts 
in the form of 
$\langle$\text{head entity}, \text{relation}, \text{tail entity}$\rangle$ triples 
(e.g. $\langle$\textit{The Matrix}, \texttt{country\_of\_film}, 
\textit{Australia}$\rangle$), which 
could support a wide range of reasoning and question answering applications. 
The Knowledge Base Completion (KBC) task aims to predict the missing part of an incomplete triple, 
such as $\langle$\textit{Finding Nemo}, \texttt{country\_of\_film}, ?$\rangle$, by reasoning 
from known facts stored in the KB.

\begin{figure}[!t]
\centering
\includegraphics[width=0.9\columnwidth]{model_arch}
\caption{In joint training, relation parameters (e.g. $\mat{M}_1$) 
receive updates from both
a \emph{KB-learning objective}, trying to predict entities in the KB; 
and 
a \emph{reconstruction objective} from an autoencoder, trying to 
recover relations from low dimension codings.}
\label{fig:model_arch}
\end{figure}


% challenges a system to recover the 
% missing part of an incomplete triple, as a test of its ability to reason from known facts. 
% For example, the system may find statistical evidence suggesting that entities joined by 
% the relation \texttt{adjoining\_country} are also likely to be 
% a tail entity of the relation \texttt{country\_of\_film}, and from the known triple 
% $\langle$\textit{Australia}, \texttt{adjoining\_country}, \textit{New Zealand}$\rangle$, 
% it may imply that \textit{New Zealand} is a probable candidate to fill 
% in the incomplete triple. 

As a most common approach \citep{DBLP:journals/tkde/WangMWG17}, modeling entities and relations to operate in a low dimension 
vector space helps KBC, for three conceivable reasons. First, when dimension is low, entities modeled 
as vectors are forced to share parameters, so ``similar'' entities which participate in 
many relations in common get close to each other (e.g. \textit{Australia} close to \textit{US}). This 
could imply that an entity (e.g. \textit{US}) ``type matches'' a relation such 
as \texttt{country\_of\_film}. Second, relations may share parameters as well, which could transfer 
facts from one relation to other similar relations, for example 
from $\langle$\textit{x}, \texttt{award\_winner}, \textit{y}$\rangle$ to 
$\langle$\textit{x}, \texttt{award\_nominated}, \textit{y}$\rangle$. Third, spatial positions might be 
used to implement \emph{composition} of relations, as relations can be regarded as mappings from head to 
tail entities, and the composition of two maps can match a third (e.g. the composition 
of \texttt{currency\_of\_country} and \texttt{country\_of\_film} 
matches the relation \texttt{currency\_of\_film\_budget}), which 
could be captured by modeling composition in a space. 

However, modeling relations as mappings naturally requires more parameters -- a general linear 
map between $d$-dimension vectors is represented by a matrix of $d^2$ parameters -- 
which are less likely to be shared, impeding transfers of facts between similar relations. 
Thus, it is desired to reduce dimensionality of relations; furthermore, 
the existence of a composition of two relations 
(assumed to be modeled by matrices $\mat{M}_1,\mat{M}_2$) matching a third ($\mat{M}_3$) also justifies 
dimension reduction, because it implies 
a \emph{compositional constraint} $\mat{M}_1\cdot \mat{M}_2\approx \mat{M}_3$ that can be 
satisfied only by a 
lower dimension sub-manifold in the parameter 
space\footnote{It is noteworthy that similar compositional constraints apply to 
most modeling schemes of relations, not just matrices.}. 

Previous approaches reduce dimensionality of relations by imposing 
pre-designed hard constraints on the parameter space, such as constraining that 
relations are translations \citep{DBLP:conf/nips/BordesUGWY13} or diagonal matrices \citep{Yang2015}, or 
assuming they are linear combinations of a small number of prototypes \citep{xie-EtAl:2017:Long}.
However, pre-designed hard constraints do not seem to cope well with compositional constraints, because 
it is difficult to know \emph{a priori} 
which two relations compose to which third relation, hence difficult to choose a pre-design; and 
compositional constraints are not always exact (e.g. the composition 
of \texttt{currency\_of\_country} and \texttt{headquarter\_location} usually matches 
\texttt{business\_operation\_currency} but not always), so hard constraints are 
less suited. 

% -- although they are likely to reside on a low dimension sub-manifold in the high dimension parameter 
% space. For one reason, composition of two matrices $M_1,M_2$ coinciding with a third 
% imposes a constraint $M_1\cdot M_2\approx M_3$ that can be satisfied only by a sub-manifold in the 
% parameter space; so we need methods to discover such constraints and reduce dimensionality. 


% Thus, assuming the relations are modeled by matrices (linear mappings) 
% $M_1$, $M_2$ and $M_3$, such a coincidence imposes a constraint 
% $M_1\cdot M_2\approx M_3$ that can only be satisfied by a sub-manifold of the parameter 
% space of all matrices. It is noteworthy that this kind of compositional constraints 
% exist in almost every modeling scheme of relations, not just for matrices models. 
% Therefore, it is often desired to reduce the dimensionality of parameters of relations.



% One elegant way of capturing such statistical evidence is to model entities 
% in a low dimensional vector space; by controlling the dimensionality, it helps driving 
% entities participating in the same relations close to each other 
% (e.g. \textit{Australia} close to \textit{New Zealand}). On the other hand, it seems that 
% modeling relations requires more parameters, because relations are 
% multi-valued mappings between entities, which have more degree of freedom 
% than the position of a single entity in a vector space. However, relations in knowledge 
% bases are also highly regularized; they 
% are likely to lie on a low dimensional sub-manifold in the whole parameter space of 
% arbitrary mappings. One compelling example comes from composition: it could happen that 
% the composition of two relations coincides with a third relation (e.g. the composition 
% of \texttt{currency\_of\_country} and \texttt{country\_of\_film} 
% reveals the relation \texttt{currency\_of\_film\_budget}). 
% Thus, assuming the relations are modeled by matrices (linear mappings) 
% $M_1$, $M_2$ and $M_3$, such a coincidence imposes a constraint 
% $M_1\cdot M_2\approx M_3$ that can only be satisfied by a sub-manifold of the parameter 
% space of all matrices. It is noteworthy that this kind of compositional constraints 
% exist in almost every modeling scheme of relations, not just for matrices models. 
% Therefore, it is often desired to reduce the dimensionality of parameters of relations.

% Previously, reduction of dimensionality has been addressed by imposing 
% pre-designed hard constraints on the parameter space, such as constraining that 
% relations are translations \citep{DBLP:conf/nips/BordesUGWY13} or diagonal matrices \citep{Yang2015}, or 
% assuming they are linear combinations of a small number of prototypes \citep{xie-EtAl:2017:Long}.
% However, in case of compositional constraints, it is difficult to know \emph{a priori} 
% which two relations would compose to which third relation, hence difficult to cope with 
% pre-designs; moreover, such constraints are not always exact (e.g. the composition 
% of \texttt{currency\_of\_country} and \texttt{headquarter\_location} usually recovers 
% \texttt{business\_operation\_currency} but not always), so hard constraints are 
% less suited. 

In this paper, we investigate an alternative approach by 
training relation parameters jointly with an autoencoder (Figure~\ref{fig:model_arch}). 
During training, the autoencoder tries to 
reconstruct relations from low dimension codings, with the 
reconstruction objective back-propagating to relation parameters 
as well. We show this novel technique promotes 
parameter sharing between different relations, and drives them toward 
low dimension manifolds (Sec.\ref{sec:analyzeautoenc}). Besides, 
we expect the technique to cope better with compositional constraints, because 
it discovers low dimension manifolds 
posteriorly from data, and it does not impose any explicit hard constraints.

Yet, joint training with an autoencoder is not simple; one has to keep a subtle balance between 
gradients of the reconstruction and KB-learning objectives throughout the training process. 
We are not aware of any theoretical principles directly addressing this problem; but we found some 
important settings after extensive pre-experiments (Sec.\ref{sec:optimizationtricks}). We evaluate our system using standard 
KBC datasets, achieving state-of-the-art on several of them (Sec.\ref{sec:mainresults}), with 
strongly improved Mean Rank. 
We discuss 
detailed settings that lead to the performance (Sec.\ref{sec:trainingbase}), 
and we show that joint training with 
an autoencoder indeed helps discovering 
compositional constraints (Sec.\ref{sec:compositionalconstraints}) and benefits from 
compositional training (Sec.\ref{sec:gainscomptrain}). 


% to dimension reduction, 

% but it does not explicitly 
% impose any hard constraint on relations; rather, as gradients of the recovering loss 
% are propagated to relations during joint training, it helps driving relations to low 
% dimensional configurations that are largely hinted by the data (Section ...). We 
% found this novel training technique leads to sparse codings of relations that are 
% interpretable (Section ...), the dimension reduction indeed helps discovering 
% compositional constraints (Section ..), and it improves performance of compositional 
% training for KBC tasks, especially in Mean Rank (Section ..). We achieve state-of-the-art 
% results on several datasets (Section ..) and discuss some crucial settings leading to 
% the results (Section ..).


% Knowledge bases (KBs), such as WordNet~\citep{DBLP:journals/cacm/Miller95},
% YAGO~\citep{DBLP:conf/www/SuchanekKW07}, and Freebase~\citep{DBLP:conf/sigmod/BollackerEPST08}
% have been created and successfully applied to many applications like
% semantic parsing~\citep{berant-EtAl:2013:EMNLP}, information extraction~\citep{DBLP:journals/pieee/Nickel0TG16},
% and question answering~\citep{Hixon2015LearningKG}.
% A KB represents relationships between entities as triples (head entity, relation, tail entity).
% Even though KBs are huge, their coverage is far from complete~\citep{DBLP:conf/naacl/MinGW0G13}.
% This fact has motivated \emph{link prediction} or \emph{knowledge base completion} (KBC),
% which is a task to complete the triple $(h, r, t)$ when one of $h, t$ is missing.
% Embedding models for KBC associate entities and relations with dense vectors or matrices,
% and achieve the state-of-the-art performances.

% この連続空間埋め込みを行う際の研究課題の一つに,関係間の知識共有がある.
% 直感的に,例えば「(人物が)(作品を)監督した」と「(人物が)(作品を)製作した」という関係はどちらも「人物」と「作品」を結びつく概念であるように,
% 多くの関係はいくつかの概念を共有するので,このような直感を知識ベースの
% モデル化に取り入れることが望ましい.
% 連続空間への埋め込みはこのような概念の共有を促すが,一方で知識
% データベースのモデル化の性質上,関係はエンティティ間を写像する演算として
% 機能しなければならないので,関係を埋め込む際の選択肢は
% 大幅に制限される.例えば,関係をエンティティベクトル間の線形変換として
% モデル化するのは自然な方法であるが,この場合に関係はエンティティベクトルの
% 次元の2乗分ものパラメータを持ち,埋め込みによる概念共有の促進作用が非常に
% 弱いと思われる.
% 逆に,関係をエンティティベクトル間の平行移動として
% モデル化する場合,関係の埋め込み次元はエンティティベクトルの次元
% と等しいが,この場合に概念共有の促進作用が非常に強いけれど,複雑な関係に
% 対しては表現力が
% 不足すると思われる\cite{DBLP:journals/corr/Nguyen17a}.このように,
% 空間埋め込みに頼るだけでは,どの選択肢も関係の概念共有に最適であるとは
% 限らない.

% It is known that relation paths, i.e., multi-hop relationships between entities give
% more performance improvement to embedding 
% models~\citep{DBLP:conf/emnlp/GuML15,DBLP:conf/emnlp/LinLLSRL15,DBLP:conf/acl/NeelakantanRM15},
% since they reflect complicated inference patterns among relations in KBs.
% For example, a relation path
% $\ent{BarackObama} - \rel{born\_in\_city} - \rel{city\_in\_state} - \rel{state\_in\_country} - \ent{UnitedStates}$
% means \rel{nationality} relation between \ent{BarackObama} and \ent{UnitedStates}.

% However, previous work,
% include \citep{DBLP:conf/emnlp/GuML15,DBLP:conf/emnlp/LinLLSRL15,DBLP:conf/acl/NeelakantanRM15},
% model only the chain of relations and do not consider the intermediate entities in relation paths.
% This can lead to poor modeling of relation paths, especially when they have 1-to-N or N-to-N relations.
% For example, a relation path
% $\ent{StevenSpielberg} - \rel{director\_of\_movie} - \rel{movie\_awarded\_for} - \ent{AcademyAwardForDirecting}$
% is lack of an important fact that only \ent{SavingPrivateRyan} was awarded for \ent{AcademyAwardForDirecting} among many movies directed by \ent{StevenSpielberg}.

% A natural solution to this problem is to lexicalize the relations ---
% i.e., replace relations such as $\rel{director\_of\_movie}$ or $\rel{movie\_awarded\_for}$ with new relations that include tail entities,
% for example $\lexrel{director\_of\_movie}{SavingPrivateRyan}$ or $\lexrel{movie\_awarded\_for}{AcademyAwardForDirecting}$.
% By using the lexicalized relations, the above example can be described as
% $\ent{StevenSpielberg} - \lexrel{director\_of\_movie}{SavingPrivateRyan} - \lexrel{movie\_awarded\_for}{AcademyAwardForDirecting} - \ent{AcademyAwardForDirecting}$,
% which means the fact $\ent{SavingPrivateRyan}$, directed by \ent{StevenSpielberg}, was awarded for \ent{AcademyAwardForDirecting}.
% However, since KBs have generally numerous types of entities and relations, the rexicalization may suffer from the data sparseness problem.

% In this paper, we propose joint training of KB embeddings and an autoencoder to reconstruct a representation of lexicalized relations.
% Since our autoencoder shares parameters across different relations, sparse relations can obtain a better representation.
% Our approach has an advantage that we can adjust a strength of parameter sharing
% without any restrictions by setting a code length of the autoencoder.

\section{Base Model}\label{sec:basemodel}

%In this section we describe a basic linear model for learning KBs. 
A knowledge base (KB) is a set $\setoftriples$ of triples of the 
form $\langle h, r, t\rangle$, where 
$h, t\in\setofents$ are entities and $r\in\setofrels$ is a relation 
(e.g. $\langle$\textit{The Matrix}, \texttt{country\_of\_film}, 
\textit{Australia}$\rangle$). A relation $r$ has its inverse 
$r^{-1}\in\setofrels$ so that for every 
$\langle h, r, t\rangle\in\setoftriples$, 
we regard $\langle t, r^{-1}, h \rangle$ as also in the KB. 
Under this assumption and given $\setoftriples$ as training data, 
we consider the Knowledge Base Completion (KBC) task that 
predicts candidates for a missing tail entity in an incomplete 
$\langle h, r, ?\rangle$ triple. 

Most approaches tackle this problem by training a \emph{score function} 
measuring the plausibility of triples being facts. 
The model we implement in this work represents entities 
$h,t$ as $d$-dimension vectors $\vec{u}_h,\vec{v}_t$ respectively, 
and relation $r$ as a $d\times d$ matrix $\mat{M}_r$. If 
$\vec{u}_h,\vec{v}_t$ are one-hot vectors with 
dimension $d=\lvert\setofents\rvert$ corresponding to each entity, 
one can take $\mat{M}_r$ as the adjacency matrix of entities joined by relation $r$, 
so the set of tail entities filling into $\langle h, r, ?\rangle$ is 
calculated by 
$\vec{u}_h^\top \mat{M}_{r}$ (with each nonzero entry corresponds to an answer). 
Thus, we have $\vec{u}_h^\top \mat{M}_{r}\vec{v}_t > 0$ if and only 
if $\langle h, r, t\rangle\in\setoftriples$. This motivates us to use 
$\vec{u}_h^\top \mat{M}_{r}\vec{v}_t$ as a natural parameter to model plausibility 
of $\langle h, r, t\rangle$, even in a 
%and we expect it to be learned using 
low dimension space with $d\ll\lvert\setofents\rvert$. 
Thus, we define the score function as 
\begin{equation}\label{eq:scrbilinear}
s(h,r,t):=\exp(\vec{u}_h^\top\mat{M}_{r}\vec{v}_t)
\end{equation}
for the basic model. This is similar to the bilinear model 
of \citet{Nickel:2011:TMC:3104482.3104584}, except that we distinguish 
$\vec{u}_h$ (the vector for head entities) from $\vec{v}_t$ (the vector for tail 
entities). It has also been proposed in \citet{tian-okazaki-inui:2016:P16-1}, but 
for modeling dependency trees rather than KBs. 

More generally, we consider \emph{composition} of 
relations $r_1/\ldots/r_l$ to model \emph{path}s in a 
KB \citep{guu-miller-liang:2015:EMNLP}, as defined by 
$r_1,\ldots,r_l$ participating in a 
sequence of facts such that the head entity of 
each fact coincides with the tail of its previous. 
For example, a sequence of two facts 
$\langle$\textit{The Matrix}, \texttt{country\_of\_film}, 
\textit{Australia}$\rangle$ and $\langle$\textit{Australia}, \texttt{currency\_of\_country}, \textit{Australian Dollar}$\rangle$ form a path of composition \texttt{country\_of\_film}\,/ \texttt{currency\_of\_country}, 
because the head of the second 
fact (i.e. \textit{Australia}) coincides with the tail of the first. 
Using the previous $d=\lvert\setofents\rvert$ analogue, one can verify 
that composition of 
relations is represented by multiplication of adjacency matrices, 
so we accordingly define 
$$
s(h,r_1/\ldots/r_l,t):=\exp(\vec{u}_h^\top\mat{M}_{r_1}\cdots\mat{M}_{r_l}\vec{v}_t)
$$
to measure the plausibility of a path. It is explored in 
\citet{guu-miller-liang:2015:EMNLP} to learn a score function not only for 
single facts but also for paths. This \emph{compositional training} scheme 
is shown to bring valuable information about the 
structure of the KB and may help KBC. In this work, we conduct experiments both with and 
without compositional training. 



% one can consider \emph{path}s \citep{DBLP:conf/emnlp/GuML15} of the 
% form $\langle h,r_1,\ldots,r_l,t\rangle$, besides triples. A score function measures 
% a path as the plausibility of reaching $t$ from $h$, by traversing a sequence of 
% relations $r_1,\ldots,r_l$ through a \emph{composition} of facts 
% (i.e., a sequence of facts with each head entity coincides with the tail of its 
% previous). For example, one 
% can reach \textit{Australian Dollar} from \textit{The Matrix}, via a composition of 
% the two facts $\langle$\textit{The Matrix}, \texttt{country\_of\_film}, 
% \textit{Australia}$\rangle$ and $\langle$\textit{Australia}, \texttt{currency\_of\_country}, 
% \textit{Australian Dollar}$\rangle$. Note that the head of the second 
% fact (i.e. \textit{Australia}) coincides with the tail of the first. It is shown 
% in \citet{DBLP:conf/emnlp/GuML15} that paths can provide valuable information 
% about structures of KBs and help KBC. In this work, we define 

% ... compositional training ...
% as the score function for paths. This model is similar to the 
% ``compositionalization'' of \citet{Nickel:2011:TMC:3104482.3104584} 
% as described in \citet{DBLP:conf/emnlp/GuML15}, except that we distinguish 
% $\vec{u}_h$ (the vector for head entities) from $\vec{v}_t$ (the vector for tail 
% entities). It has also been proposed in \citet{tian-okazaki-inui:2016}, but for 
% modeling dependency trees rather than KBs. 

In order to learn parameters $\vec{u}_h,\vec{v}_t,\mat{M}_r$ of the score 
function, we 
follow \citet{tian-okazaki-inui:2016:P16-1} using a 
Noise Contrastive Estimation (NCE) \citep{DBLP:journals/jmlr/GutmannH12} objective. For each 
path (or triple) $\langle h,r_1/\ldots,t\rangle$ taken from the KB, 
we generate negative samples by replacing the tail entity $t$ with some 
random noise 
$t^{*}$. Then, we maximize 
\begin{multline*}
\mathcal{L}_1:=
\sum_{\text{path}}\ln\frac{s(h, r_1/\ldots, t)}{k+s(h, r_1/\ldots, t)}\\
+\sum_{\text{noise}}\ln\frac{k}{k+s(h, r_1/\ldots, t^{*})}
\end{multline*}
as our \emph{KB-learning objective}. Here, $k$ is the number of noises 
generated for each path. When the score function 
is regarded as probability, $\mathcal{L}_1$ represents 
the log-likelihood of 
``$\langle h,r_1/\ldots,t\rangle$ being actual path and 
$\langle h,r_1/\ldots,t^{*}\rangle$ being noise''. Maximizing $\mathcal{L}_1$ 
increases the scores of actual paths and decreases the scores of noises.

\section{Joint Training with an Autoencoder}
\label{sec:jointtraining}

% \subsection{Knowledge Base Embedding Model}

% Our KB embedding model is based on the \emph{bilinear comp} \citep{DBLP:conf/emnlp/GuML15}
% and the training scheme of vector-based DCS \citep{tian-okazaki-inui:2016}.

% 知識ベース補完のモデルとして,我々はGuuらのパス情報を取り入れた
% 双線型モデル\cite{DBLP:conf/emnlp/GuML15}とvecDCS\cite{tian-okazaki-inui:2016}の訓練手法をベースにする.この
% モデルでは,エンティティを$d$-次元のベクトル,関係を$(d\times d)$の行列として表現する.
% 学習は,一つのエンティティ$h$から出発し,いくつかの
% 関係$r_1,\ldots,r_n$からなるパスを経由して辿り着いたもう一つの
% エンティティ$t$に対して,エネルギー関数$f(h, r_1,\ldots,r_n, t)$を最大化することで
% エンティティベクトル$\vec{h},\;\vec{t}$と
% 行列$\mat{M}_{r_1},\ldots,\mat{M}_{r_n}$を推定する.
% エネルギー関数は
% \begin{equation}
% f(h, r_1,\ldots,r_n, t) := \exp(^\top\vec{h}\mat{M}_{r_1},\ldots,\mat{M}_{r_n}\vec{t})
% \end{equation}
% と定義し,知識ベースからこのようなデータが取れる尤度に相当する.
% 推定時は,知識ベースからのデータと合わせ,ランダムに生成された$k$個の負例$h, r'_1,\ldots,r'_n, t'$を使って
% \begin{equation}\label{eq:main}
% \frac{f(h, r_1,\ldots,r_n, t)}{k+f(h, r_1,\ldots,r_n, t)}
% \cdot\prod_{k}\frac{k}{k+f(h, r'_1,\ldots,r'_n, t')}
% \end{equation}
% を最大化する.これは,「$(h, r_1,\ldots,r_n, t)$が正例で$(h, r'_1,\ldots,r'_n, t')$が負例」であるイベントの尤度に相当する.

% \subsection{Joint training with an autoencoder}

Autoencoders learn efficient codings of high-dimensional data while trying to 
reconstruct the original data from the coding. By joint training relation matrices 
with an autoencoder, we also expect it to 
help reducing the dimensionality of the original data (i.e. relation matrices). 

Formally, we define a \emph{vectorization} $\vec{m}_r$ for each relation matrix 
$\mat{M}_r$, and use it as input to the autoencoder. $\vec{m}_r$ is defined as a reshape 
of $\mat{M}_r$ flattened into a $d^2$-dimension vector, and normalized such 
that $\lVert\vec{m}_r\rVert=\sqrt{d}$. We define 
\begin{equation}\label{eq:coding}
\vec{c}_r:=\relu(\mat{A}\vec{m}_{r})
\end{equation}
as the coding. Here $\mat{A}$ is a $c\times d^2$ matrix with $c\ll d^2$, and 
$\relu$ is the Rectified Linear Unit function \citep{DBLP:conf/icml/NairH10}. 
We reconstruct the input from $\vec{c}_r$ by multiplying a 
$d^2\times c$ matrix $\mat{B}$. We want $\mat{B}\vec{c}_{r}$ to be more similar 
to $\vec{m}_r$ than other relations. For this purpose, we define a similarity  
\begin{equation}
\label{eq:reconscore}
g(r_1, r_2):=\exp(\frac{1}{\sqrt{dc}}\vec{m}_{r_1}^\top\mat{B}\vec{c}_{r_2}), 
\end{equation}
which measures the length of $\mat{B}\vec{c}_{r_2}$ projected to the direction 
of $\vec{m}_{r_1}$. In order to 
learn the parameters $\mat{A},\mat{B}$, we adopt the Noise Contrastive Estimation scheme as in Sec.\ref{sec:basemodel}, 
generate random noises $r^{*}$ for each relation $r$ and maximize 
$$
\mathcal{L}_2:=
\sum_{r\in\setofrels}\ln\frac{g(r, r)}{k+g(r, r)}
+\sum_{r^{*}\sim\setofrels}\ln\frac{k}{k+g(r, r^{*})}
$$
as our \emph{reconstruction objective}. Maximizing $\mathcal{L}_2$ increases 
$\vec{m}_r$'s similarity with $\mat{B}\vec{c}_{r}$, and decreases it with 
$\mat{B}\vec{c}_{r^{*}}$. 

%The factor $\frac{1}{\sqrt{dc}}$ in $\eqref{eq:reconscore}$ is crucial for 
%decent training, as we will discuss in Sec....

During joint training, both $\mathcal{L}_1$ and $\mathcal{L}_2$ are 
simultaneously maximized, and the gradient $\nabla\mathcal{L}_2$ propagates to 
relation matrices as well. Since $\nabla\mathcal{L}_2$ depends on $\mat{A}$ 
and $\mat{B}$, and 
$\mat{A},\mat{B}$ interact with all relations, they promote indirect 
parameter sharing between different relation matrices. 
In Sec.\ref{sec:analyzeautoenc}, we further 
show that joint training drives relations toward a low dimension manifold.


% Our autoencoder is designed to reconstruct projection matrices of relations.
% Let $\vec{m}_r$ be a $d^2$-dimensional vector
% flattened a projection matrix of relations $\mat{M}_r$.
% The autoencoder transforms $\vec{m}_r$ into a $l$-dimensional code vector ($l \ll d^2$)
% by a matrix $\mat{A} \in \mathbb{R}^{d^2 \times l}$,
% then applies a nonlinear function $\relu$ to the code vector $\mat{A}\vec{m}_r$,
% and reconstructs the input vector by a matrix $\mat{B} \in \mathbb{R}^{l \times d^2}$:
% \begin{equation}
% \vec{m}_{r}\approx\mat{B}\relu(\mat{A}\vec{m}_{r})
% \end{equation}
% We define a scoring function of the autoencoder to maximize the similarity
% between $\vec{m}_r$ and the reconstructed vector as follows.
% \begin{equation}
% g(\vec{m}_{r}):=\exp(\vec{m}_{r}\cdot\mat{B}\relu(\mat{A}\vec{m}_{r}))
% \end{equation}
% を定義し,最適化の際にランダムに生成された$k$個の負例$\vec{m}_{r'}$と合わせて
% \begin{equation}\label{eq:autoencoder}
% \frac{g(\vec{m}_{r})}{k+g(\vec{m}_{r})}\cdot\prod_{k}\frac{k}{k+g(\vec{m}_{r'})}
% \end{equation}
% を最大化する.また,式\eqref{eq:autoencoder}の最大化において$\vec{m}_{r}$に対する
% 勾配も計算し,これと式\eqref{eq:main}で計算された$\vec{m}_{r}$の勾配と合わせて
% パラメータ$\vec{m}_{r}$の更新を行う.オートエンコーダ
% との同時学習によって,$\vec{m}_{r}$が低次元のコードから「復元されやすい位置」,つまり
% 類似した関係同士がクラスタしているような空間位置に動くと期待される。また,全ての関係$r$に
% 対して$\vec{m}_{r}$が同じ行列$\mathbf{A},\;\mathbf{B}$によってエンコード・デコード
% されるので,異なる$\vec{m}_{r}$同士が行列$\mathbf{A},\;\mathbf{B}$を介してパラメータを共有しているとの見方もできる.これによって,異なる関係間の知識共有が促されると思われる.

\section{Optimization Tricks}\label{sec:optimizationtricks}

Joint training with an autoencoder is not simple. Relation matrices receive updates 
from both $\nabla\mathcal{L}_1$ and $\nabla\mathcal{L}_2$, but if they update 
$\nabla\mathcal{L}_1$ too much, the autoencoder has no effect; conversely, if they 
update $\nabla\mathcal{L}_2$ too often, all relation matrices crush into one 
cluster. Furthermore, an autoencoder should learn from 
genuine patterns of relation matrices that emerge from fitting the KB, but not 
the reverse -- in which the autoencoder imposes arbitrary patterns to 
relation matrices 
according to random initialization. Therefore, it is not surprising that 
a naive optimization of 
$\mathcal{L}_1+\mathcal{L}_2$ does not work. 

After extensive pre-experiments, we have found some crucial settings for successful 
training. The most important ``magic'' is the scaling factor $\frac{1}{\sqrt{dc}}$ 
in definition of the similarity function \eqref{eq:reconscore}, perhaps being 
combined with other settings as we discuss below. We have tried different 
factors $1$, $\frac{1}{\sqrt{d}}$, $\frac{1}{\sqrt{c}}$ and 
$\frac{1}{dc}$ instead, with various combinations of $d$ and $c$; but 
the autoencoder failed to learn meaningful codings in other settings. 
When the scaling factor is too small 
(e.g. $\frac{1}{dc}$), all relations get almost the same coding; conversely if 
the factor is too large (e.g. $1$), all codings get very close to $0$. 

The next important rule is to keep a balance between the updates coming from 
$\nabla\mathcal{L}_1$ and $\nabla\mathcal{L}_2$. We 
use Stochastic Gradient 
Descent (SGD) for optimization, and the common practice \citep{bottou2012stochastic} is to set the 
learning rate as
\begin{equation}\label{eq:commonpractice}
\alpha(\tau):=\frac{\eta}{1+\eta\lambda\tau}. 
\end{equation}
Here, $\eta,\lambda$ are hyper-parameters and $\tau$ is a counter of 
processed data points. In this work, in order to control 
the updates in detail to keep a balance, we modify \eqref{eq:commonpractice} to use a 
a step 
counter $\tau_r$ for each relation $r$, counting ``number of updates'' 
instead of 
data points\footnote{Similarly, we set separate step counters for all head 
and tail entities, and the autoencoder as well.}. That is, whenever $\mat{M}_r$ gets a 
nonzero update from a gradient calculation, $\tau_r$ increases by $1$. 
Furthermore, we use different hyper-parameters for different ``types of updates'', 
namely $\eta_1,\lambda_1$ for updates coming from $\nabla\mathcal{L}_1$, and 
$\eta_2,\lambda_2$ for updates coming from $\nabla\mathcal{L}_2$. 
Thus, let $\Delta_1$ be the partial gradient of $\nabla\mathcal{L}_1$, and $\Delta_2$ the partial gradient of $\nabla\mathcal{L}_2$, we 
update $\mat{M}_r$ by $\alpha_1(\tau_r)\Delta_1+\alpha_2(\tau_r)\Delta_2$ at 
each step, where 
$$
\alpha_1(\tau_r):=\frac{\eta_1}{1+\eta_1\lambda_1\tau_r},\;\;
\alpha_2(\tau_r):=\frac{\eta_2}{1+\eta_2\lambda_2\tau_r}.
$$

The rule for setting $\eta_1,\lambda_1$ and $\eta_2,\lambda_2$ 
is that, $\eta_2$ should be much smaller than $\eta_1$, because 
$\eta_1,\eta_2$ control the magnitude of learning rates at the early stage of 
training, with the autoencoder still largely random and $\Delta_2$ not 
making much sense; on the other hand, one has to choose $\lambda_1$ and 
$\lambda_2$ such that 
$\lVert\Delta_1\rVert/\lambda_1$ and $\lVert\Delta_2\rVert/\lambda_2$ are at 
the same scale, because the learning rates approach $1/(\lambda_1\tau_r)$ 
and $1/(\lambda_2\tau_r)$ respectively, as the training proceeds. 
In this way, the autoencoder will not impose random patterns to relation matrices 
according to its initialization at the early stage, and a balance 
is kept between $\alpha_1(\tau_r)\Delta_1$ and $\alpha_2(\tau_r)\Delta_2$ later.

But how to estimate $\lVert\Delta_1\rVert$ and $\lVert\Delta_2\rVert$? It seems 
that we can approximately calculate their scales from initialization. In this 
work, we use i.i.d. Gaussians of variance $1/d$ to initialize parameters, 
so the initial Euclidean norms are $\lVert\vec{u}_h\rVert\approx 1$, 
$\lVert\vec{v}_t\rVert\approx 1$, 
$\lVert\mat{M}_r\rVert\approx\sqrt{d}$, and 
$\lVert\mat{B}\mat{A}\vec{m}_r\rVert\approx\sqrt{dc}$.
%\footnote{We use the Frobenius norm for matrices.}
Thus, by calculating $\nabla\mathcal{L}_1$ and $\nabla\mathcal{L}_2$ using 
\eqref{eq:scrbilinear} and \eqref{eq:reconscore}, we have approximately 
\begin{gather}
\lVert\Delta_1\rVert\approx\lVert\vec{u}_h\vec{v}_t^\top\rVert\approx 1, 
\quad\text{and}\\
\lVert\Delta_2\rVert\approx\lVert\frac{1}{\sqrt{dc}}\mat{B}\vec{c}_r\rVert\approx
\frac{1}{\sqrt{dc}}\lVert\mat{B}\mat{A}\vec{m}_r\rVert\approx 1.
\end{gather}
It suggests that, because of the scaling factor $\frac{1}{\sqrt{dc}}$ in 
\eqref{eq:reconscore}, we have 
$\lVert\Delta_1\rVert$ and $\lVert\Delta_2\rVert$ at the same scale, so we can 
set $\lambda_1=\lambda_2$. This might not be a mere coincidence. 

% In our experiments, the factor $\frac{1}{\sqrt{dc}}$ turns out to be crucial; we 
% have tried using factors $1$, $\frac{1}{\sqrt{d}}$, $\frac{1}{\sqrt{c}}$ and 
% $\frac{1}{dc}$ instead, 
% but although we can achieve balanced joint training each time by adjusting 
% $\lambda_1$ and $\lambda_2$, the autoencoder failed to 
% learn meaningful codings after all. When the factor is too small 
% (e.g. $\frac{1}{dc}$), all relations get almost the same coding; conversely if 
% the factor is too large (e.g. $1$), all codings get very close to $0$. 

% maybe for discussion ???
% This modification of SGD has 
% traits similar to some modern optimization algorithms such as Adagrad \citep{..}, 
% in that they both set different learning rates for different parameters. While 
% Adagrad sets them adaptively by keeping track of past gradients for 
% all parameters, our modification of SGD is more efficient and allows us to 
% grasp a rough intuition about which parameter gets how much update. 

\subsection{Training the Base Model}\label{sec:trainingbase}

Besides the tricks for joint training, we also found settings that 
significantly improve the base model on KBC, as briefly 
discussed below. In Sec.\ref{sec:crucialsettings}, we will show 
performance gains by these settings using the FB15k-237 validation set.

\paragraph{Normalization} It is better to normalize 
relation matrices to $\lVert\mat{M}_r\rVert=\sqrt{d}$ during training. 
This might reduce 
fluctuations in entity vector updates. 

\paragraph{Regularizer} It is better to minimize 
$\lVert \mat{M}_r^\top \mat{M}_r-\frac{1}{d}\tr(\mat{M}_r^\top \mat{M}_r)I\rVert$ during training. 
This regularizer drives $\mat{M}_r$ toward an orthogonal matrix 
\citep{tian-okazaki-inui:2016:P16-1} and might reduce 
fluctuations in entity vector updates. As a result, all relation matrices trained 
in this work are very close to orthogonal.

\paragraph{Initialization} Instead of pure Gaussian, it is better to initialize 
matrices as $(I+G)/2$, where $G$ is random. The identity matrix $I$ helps passing information from 
head to tail \citep{tian-okazaki-inui:2016:P16-1}.

\paragraph{Negative Sampling} Instead of a unigram distribution, it is better 
to use a \emph{uniform} distribution for generating noises. This is 
somehow counter-intuitive compared to training word embeddings. 

%\subsection{Training of Autoencoder}
\section{Related Works}

KBs have a wide range of applications
\citep{berant-EtAl:2013:EMNLP,hixon-clark-hajishirzi:2015:NAACL-HLT,DBLP:journals/pieee/Nickel0TG16}
and KBC has inspired a huge amount of research 
\citep{DBLP:conf/nips/BordesUGWY13,riedel-EtAl:2013:NAACL-HLT,DBLP:conf/nips/SocherCMN13,DBLP:conf/aaai/WangZFC14,wang-EtAl:2014:EMNLP20145,DBLP:conf/ijcai/xiao16,nguyen-EtAl:2016:N16-1,
toutanova-EtAl:2016:P16-1,das-EtAl:2017:EACLlong1,hayashi-shimbo:2017:Short}. 

Among the previous works, TransE \cite{DBLP:conf/nips/BordesUGWY13} is the 
classic method which represents a relation as a translation of the entity 
vector space, and is partially inspired by \citet{mikolov-yih-zweig:2013:NAACL-HLT}'s vector arithmetic method of 
solving word analogy tasks. Although competitive in KBC, it is speculated that 
this method is well-suited for $1$-to-$1$ relations but might be too 
simple to represent $N$-to-$N$ relations 
accurately\cite{DBLP:journals/tkde/WangMWG17}. Thus, extensions such as 
TransR \cite{DBLP:conf/aaai/LinLSLZ15} and STransE \cite{nguyen-EtAl:2016:N16-1} 
are proposed to map entities into a relation-specific vector space before 
translation. The ITransF model \cite{xie-EtAl:2017:Long} further enhances this 
approach by imposing a hard constraint that the relation-specific maps should be 
linear combinations of a small number of prototypical matrices. Our work 
inherits the same motivation with ITransF in terms of 
promoting parameter-sharing among relations. 

On the other hand, the base model used in this work originates from 
RESCAL \cite{Nickel:2011:TMC:3104482.3104584}, in which relations are 
naturally represented as analogue to the adjacency matrices 
(Sec.\ref{sec:basemodel}). Further 
developments include HolE \cite{DBLP:conf/aaai/NickelRP16} and 
ConvE \cite{dettmers2018conve} which improve this approach in terms of 
parameter-efficiency, by introducing low dimension factorizations of the 
matrices. We inherit the basic model of RESCAL but draw 
additional training techniques from \citet{tian-okazaki-inui:2016:P16-1}, and 
show that the base model already can achieve near 
state-of-the-art performance (Sec.\ref{sec:mainresults},\ref{sec:crucialsettings}). This sends a message 
similar to \citet{kadlec-bajgar-kleindienst:2017:RepL4NLP}, saying that 
training tricks might be as important as model designs.

Nevertheless, we emphasize the novelty of this work in that the previous 
models mostly achieve dimension reduction by imposing some pre-designed 
hard constraints \citep{DBLP:conf/nips/BordesUGWY13,Yang2015,DBLP:conf/icml/TrouillonWRGB16,DBLP:conf/aaai/NickelRP16,xie-EtAl:2017:Long,dettmers2018conve}, whereas the 
constraints themselves are not learned from data; in contrast, 
our approach by jointly training an autoencoder does not impose 
any explicit hard constraints, so it leads to more flexible modeling. 

Moreover, we additionally focus on leveraging composition in KBC. Although 
this idea has been frequently explored 
before \citep{guu-miller-liang:2015:EMNLP,neelakantan-roth-mccallum:2015:ACL-IJCNLP,lin-EtAl:2015:EMNLP1}, our discussion about the concept of 
compositional constraints and its connection to dimension reduction 
has not been addressed similarly in previous research. In experiments, we will 
show (Sec.\ref{sec:compositionalconstraints},\ref{sec:gainscomptrain}) 
that joint training 
with an autoencoder indeed helps finding compositional constraints and 
benefits from compositional training.



%In this work, we achieved strong performance with a very simple base 
%model similar to \citet{Nickel:2011:TMC:3104482.3104584}, by applying 
%some detailed but crucial settings. This sends a message similar to 
%\citet{kadlec-bajgar-kleindienst:2017:RepL4NLP}.

%Nevertheless, we have focused on composition, an idea that has been 
%explored frequently 
%\citep{guu-miller-liang:2015:EMNLP,neelakantan-roth-mccallum:2015:ACL-IJCNLP,lin-EtAl:2015:EMNLP1}, but our discussion about the concept of 
%compositional constraints and its connection to dimension reduction 
%has not been addressed similarly in previous research. 

%Dimension reduction is commonly addressed by KB embedding models, mostly with 
%a pre-designed low dimension representation \citep{DBLP:conf/nips/BordesUGWY13,Yang2015,DBLP:conf/icml/TrouillonWRGB16}, 
%or factorization into 
%low dimension spaces \citep{DBLP:conf/aaai/NickelRP16,dettmers2018conve}. The paradigm of learning such dimension reduction 
%from data has been pursued by \citet{xie-EtAl:2017:Long}, but they still impose hard constraints 
%on the parameter space. In contrast, joint training with an autoencoder does not impose 
%any explicit constraints, which could lead to more flexible modeling. 

Autoencoders have been used solo for learning distributed representations 
of syntactic trees \citep{socher-EtAl:2011:EMNLP}, words and 
images \citep{silberer-lapata:2014:P14-1}, 
or semantic roles \citep{titov-khoddam:2015:NAACL-HLT}. It is also used 
for pretraining other deep neural networks \citep{Erhan:2010:WUP}. 
However, when combined with other models, the learning of autoencoders, or more generally 
\emph{sparse codings} \citep{rubinstein2010dictionaries}, is 
usually conveyed in an alternating manner, %by alternatively , 
fixing one part of the model while optimizing the other, 
such as in \citet{xie-EtAl:2017:Long}. 
To our knowledge, joint training with an autoencoder is not widely used previously 
for reducing dimensionality. 

Jointly training an autoencoder is not simple because it takes 
non-stationary inputs. 
In this work, we modified SGD so that it shares 
%have rearranged some common practice of SGD, which has 
traits with some modern optimization algorithms such as Adagrad \citep{DBLP:journals/jmlr/DuchiHS11}, 
in that they both set different learning rates for different parameters. While 
Adagrad sets them adaptively by keeping track of gradients for 
all parameters, our modification of SGD is more efficient and allows us to 
grasp a rough intuition about which parameter gets how much update. 
We believe our techniques and findings in joint training with an autoencoder 
could be helpful to reducing dimensionality and improving interpretability in 
other neural network architectures as well. 

\section{Experiments}

We evaluate on standard KBC datasets, 
%For evaluating our proposed model,
%we use a selection of standard KBC datasets from the literature:
including WN18 and FB15k \citep{DBLP:conf/nips/BordesUGWY13}, 
WN18RR \citep{dettmers2018conve} and FB15k-237 \citep{toutanova-chen:2015:CVSC}.
%WN18 \citep{DBLP:conf/nips/BordesUGWY13},
%FB15k \citep{DBLP:conf/nips/BordesUGWY13},
%WN18RR \citep{dettmers2018conve}, and
%FB15k-237 \citep{Toutanova2015a}.
The statistical information of these datasets are shown 
in Table~\ref{tab:datasets}.

\begin{table}[t]
\centering
\setlength{\tabcolsep}{4pt}
\small
\begin{tabular}{lrrrrr}
\toprule
Dataset & \multicolumn{1}{c}{$\lvert\setofents\rvert$} & \multicolumn{1}{c}{$\lvert\setofrels\rvert$} & \multicolumn{1}{c}{\#Train} & \multicolumn{1}{c}{\#Valid} & \multicolumn{1}{c}{\#Test} \\
\midrule
WN18 & 40,943 & 18 & 141,442 & 5,000 & 5,000 \\
FB15k & 14,951 & 1,345 & 483,142 & 50,000 & 59,071 \\
WN18RR & 40,943 & 11 & 86,835 & 3,034 & 3,134 \\
FB15k-237 & 14,541 & 237 & 272,115 & 17,535 & 20,466 \\
%YAGO3-10 & 123,182 & 37 & 1,079,040 & 5,000 & 5,000 \\
%Countries S1 & 271 & 2 & 1,111 & 24 & 24 \\
%Countries S2 & 271 & 2 & 1,063 & 24 & 24 \\
%Countries S3 & 271 & 2 & 985 & 24 & 24 \\
%Nations & 70 & 14 & 1,619 & 202 & 203 \\
%UMLS & 135 & 46 & 5216 & 652 & 661 \\
%Kinship & 129 & 104 & 8,548 & 1,069 & 1,069 \\
\bottomrule
\end{tabular}
\caption{%
Statistical information of the KBC datasets.
$\lvert\setofents\rvert$ and $\lvert\setofrels\rvert$ denote the number of 
entities and relation types, respectively; 
\#Train, \#Valid, and \#Test are the numbers of triples
in the training, validation, and test sets, respectively.}
\label{tab:datasets}
\end{table}

WN18 collects word relations from 
WordNet \citep{DBLP:journals/cacm/Miller95}, and FB15k is taken from 
Freebase \citep{DBLP:conf/sigmod/BollackerEPST08}; both have filtered out low 
frequency entities. However, it is reported in \citet{toutanova-chen:2015:CVSC} that 
both WN18 and FB15k have information leaks because 
the inverses of some test triples appear in the training set. 
FB15k-237 and WN18RR fix this problem by deleting such triples from training 
and test data. In this work, we do evaluate on WN18 and FB15k, but our models 
are mainly tuned on FB15k-237. 

For all datasets, we set the dimension $d=256$ and $c=16$, the SGD 
hyper-parameters $\eta_1=1/64$, 
$\eta_2=2^{-14}$ and $\lambda_1=\lambda_2=2^{-14}$. The training batch size 
is 32 and the triples in each batch share the same head entity. 
We compare the base model (\textsc{base}) to our joint training 
with an autoencoder model (\textsc{joint}), and the base 
model with compositional training (\textsc{base+comp}) to our
joint model with compositional training (\textsc{joint+comp}). 
When compositional training is 
enabled (\textsc{base+comp}, \textsc{joint+comp}), 
we use random walk to sample paths of length $1+X$, where $X$ is drawn from a Poisson 
distribution of mean $\lambda=1.0$.

% We note that \citet{Toutanova2015a} have reported that
% WN18 and FB15k have an information leak between the training and test sets.
% Although we evaluate our model on these datasets to compare with more previous methods,
% we conduct a detailed analysis of our method on FB15k-237 dataset that do not have such leak.

For any incomplete triple $\mtriple{h}{r}{?}$ in KBC test, we calculate 
a score $s(h,r,e)$ from \eqref{eq:scrbilinear}, for 
every entity $e\in\setofents$ such that $\mtriple{h}{r}{e}$ 
\emph{does not appear in any of the training, validation, or test sets}
\citep{DBLP:conf/nips/BordesUGWY13}. 
Then, the calculated scores together with $s(h,r,t)$ for the gold triple 
is converted to ranks, and the rank of the gold entity $t$ is used for 
evaluation. Evaluation metrics include Mean Rank (MR),
Mean Reciprocal Rank (MRR), and
Hits at 10 (H10). 
Lower MR, higher MRR, and higher H10 indicate better performance.

We consult MR and MRR on validation sets to determine training epochs; we stop 
training when both MR and MRR have stopped improving. 


%evaluation metrics
% The knowledge base completion is a task
% to predict the head or the tail entity given the relation and the other entity,
% i.e., predict $h$ given $\mtriple{?}{r}{t}$ or predict $t$ given $\mtriple{h}{r}{?}$.
% Specifically, we use \emph{filtered} evaluation protocol proposed by \citet{DBLP:conf/nips/BordesUGWY13}.
% For all test triples $\mtriple{h}{r}{t}$,
% (1) we calculate scores $s(e, r, t)$ for all triples $\mtriple{e}{r}{t}$
% s.t. $e \in \setofents$ and $\mtriple{e}{r}{t} \not\in \setoftriples \setminus \{\mtriple{h}{r}{t}\}$,\footnote{%
% This is to avoid penalizing the model for ranking other correct triples higher than the testing triple.}
% (2) we sort values by decreasing order, and
% (3) we record the rank of correct triple $\mtriple{h}{r}{t}$.
% A same process is repeated for predicting $t$.
% We report
% the mean of those predicted ranks (MR),
% the mean of reciprocal ranks (MRR), and
% the proportion of correct entities ranked in the top $k$ (Hits@$k$).
% Lower MR, higher MRR, or higher Hits@$k$ mean better performance.

\subsection{KBC Results}\label{sec:mainresults}


\begin{figure}[!t]
\centering
\includegraphics[width=\columnwidth]{code_heat}
\caption{%
Examples of relation codings learned from FB15k-237. Each row shows 
a 16 dimension vector encoding a relation. Vectors are normalized 
such that their entries sum to $1$.}
\label{fig:code-heatmap}
\end{figure}



\begin{table*}[!t]
\centering
\setlength{\tabcolsep}{5pt}
\small
\begin{tabular}{@{}lcccccccccc@{}}
\toprule
\multirow{2}{*}{Model} & \multicolumn{2}{c}{WN18} & \multicolumn{2}{c}{FB15k} & \multicolumn{3}{c}{WN18RR} & \multicolumn{3}{c}{FB15k-237} \\
\cmidrule(lr){2-3} \cmidrule(lr){4-5} \cmidrule(lr){6-8} \cmidrule(l){9-11}
 & MR & H10 & MR & H10 & MR & MRR & H10 & MR & MRR & H10 \\
\midrule
%\textsc{joint} & 279 & 95.8 & 53 & 81.9 & \makecell{3514 \\ (3253)} & \makecell{\textbf{.464} \\ (\textbf{.484})} & \makecell{\textbf{55.8} \\ (\textbf{56.7})} & 212 & .336 & \textbf{52.3} \\
%\textsc{base} & 288 & 95.7 & 52 & 82.0 & \makecell{3500 \\ (3193)} & \makecell{.462 \\ (.484)} & \makecell{55.7 \\ (56.6)} & 215 & \textbf{.337} & \textbf{52.3} \\
\textsc{joint} & \textbf{277} & \textbf{95.8} & \textbf{53} & \textbf{82.5} & \textbf{4233} & \textbf{.461}$^*$ & \textbf{53.4} & \textbf{212} & .336 & \textbf{52.3}$^*$ \\
\textsc{base} & 286 & \textbf{95.8} & \textbf{53} & \textbf{82.5} & 4371 & .459 & 52.9 & 215 & \textbf{.337}$^*$ & \textbf{52.3}$^*$ \\
\midrule
%\textsc{joint+comp} & \textbf{191} & 94.8 & 61 & 74.2 & \makecell{2293 \\ (1906)} & \makecell{.286 \\ (.305)} & \makecell{52.2 \\ (55.8)} & \textbf{197} & .331 & 51.6 \\
%\textsc{base+comp} & 195 & 94.8 & 65 & 74.3 & \makecell{\textbf{2214} \\ (\textbf{1853})} & \makecell{.284 \\ (.304)} & \makecell{52.0 \\ (55.6)} & 203 & .328 & 51.5 \\
\textsc{joint+comp} & \textbf{191}$^*$ & \textbf{94.8} & \textbf{53} & \textbf{69.7} & \textbf{2268}$^*$ & \textbf{.343} & \textbf{54.8}$^*$ & \textbf{197}$^*$ & \textbf{.331} & \textbf{51.6} \\
\textsc{base+comp} & 195 & \textbf{94.8} & 54 & 69.4 & 2447 & .310 & 54.1 & 203 & .328 & 51.5 \\
\midrule
%TransE & 433 & 94.3 & 63 & 64.0 & 5749 & .17 & 40.8 & 234 & .27 & 44.6 \\
TransE \citep{DBLP:conf/nips/BordesUGWY13} & 292 & 92.0 & \textbf{66} & 70.4 & 4311 & .202 & 45.6 & \textbf{278} & .236 & 41.6 \\
%TransR \citep{DBLP:conf/aaai/LinLSLZ15} & \textbf{281} & 93.6 & 77 & 68.7 & \textbf{4222} & .210 & \textbf{47.1} & 320 & \textbf{.282} & \textbf{45.9} \\ % ACL2018 submission
TransR \citep{DBLP:conf/aaai/LinLSLZ15} & \textbf{281} & 93.6 & 76 & \textbf{74.4} & \textbf{4222} & .210 & \textbf{47.1} & 320 & \textbf{.282} & \textbf{45.9} \\
RESCAL \citep{Nickel:2011:TMC:3104482.3104584} & 911 & 58.0 & 163 & 41.0 & 9689 & .105 & 20.3 & 457 & .178 & 31.9 \\
HolE \citep{DBLP:conf/aaai/NickelRP16} & 724 & \textbf{94.3} & 293 & 66.8 & 8096 & \textbf{.376} & 40.0 & 1172 & .169 & 30.9 \\
\midrule
%TransE & 251 & 89.2 & 125 & 47.1 & - & - & - & - & - & - \\
%TransH & 303 & - & 86.7 & 87 & - & 64.4 & - & - & - & &&& - & - & - \\
%TransR & 225 & -     & 92.0   & -      & -      & 77  & -     & 68.7   & -      & -      \\
STransE \citep{nguyen-EtAl:2016:N16-1} & 206 & 93.4 & 69 & 79.9 & - & - & - & - & - & - \\
ITransF \citep{xie-EtAl:2017:Long} & \textbf{205} & 94.2 & 65 & 81.0 & - & - & - & - & - & - \\
%HolE & - & 94.9 & - & 73.9 & - & - & - & - & - & - \\
%RESCAL$^\dagger$ & - & 92.8 & - & 58.7 & - & - & - & - & - & - \\
%DistMult$^\ddagger$ & 902 & 93.6 & 97 & 82.4 & 5110 & .43 & 49 & 254 & .241 & 41.9 \\
ComplEx \citep{DBLP:conf/icml/TrouillonWRGB16} & - & 94.7 & - & 84.0 & \textbf{5261} & .44 & \textbf{51} & 339 & .247 & 42.8 \\
%\texttt{fastText} & - & 94.9 & - & 86.5 & - & - & - & - & - & 44.8 \\
%Single DistMult & 655 & 94.6 & 42.2 & 89.3 & - & - & - & - & - & - \\
Ensemble DistMult \citep{kadlec-bajgar-kleindienst:2017:RepL4NLP} & 457 & 95.0 & 35.9 & 90.4 & - & - & - & - & - & - \\
IRN \citep{shen-EtAl:2017:RepL4NLP1} & 249 & 95.3 & 38 & \textbf{92.7$^*$} & - & - & - & - & - & - \\
ConvE \citep{dettmers2018conve} & 504 & 95.5 & 64 & 87.3 & 5277 & \textbf{.46} & 48 & \textbf{246} & \textbf{.316} & \textbf{49.1} \\
%Inverse Model & 567 & 96.9 & 1897 & 73.7 & - & - & - & - & - & - \\
R-GCN+ \citep{DBLP:journals/corr/SchlichtkrullKB17} & - & \textbf{96.4}$^*$ & - & 84.2 & - & - & - & - & .249 & 41.7 \\
%\texttt{fastText} - train+valid & - & \textbf{97.6} & - & 89.9 & - & - & - & - & - & 45.8 \\
ProjE \citep{DBLP:conf/aaai/ShiW17} & - & - & \textbf{34$^*$} & 88.4 & - & - & - & - & - & - \\
\bottomrule
\end{tabular}
\caption{%
% Knowledge base completion results on WN18, FB15k, WN18RR, and FB15k-237.
% H10 stands for Hits@10.
% Results mark (${}^\dagger$) and (${}^\ddagger$) taken from
% \citet{DBLP:conf/aaai/NickelRP16} and \citet{dettmers2018conve}, respectively.
% The number in the brackets denote the result of ignoring test triples that contain OOV entities.
% The first and second block represents our models with and without compositional training, respectively.
% The third block includes previous models.
KBC results on the WN18, FB15k, WN18RR, and FB15k-237 datasets. The first and 
second sectors compare our joint to the base models with and without compositional 
training, respectively; the third sector shows our re-experiments and the fourth 
shows previous published results. Bold numbers are the best in each sector, and $(^*)$ indicates 
the best of all.}
\label{tab:main-results}
\end{table*}

The results are shown in Table~\ref{tab:main-results}. We found that joint 
training with an autoencoder mostly improves performance, and 
the improvement becomes more clear when compositional training 
is enabled (i.e., $\textsc{joint}\geq\textsc{base}$ and $\textsc{joint+comp}>\textsc{base+comp}$). This is convincing because 
generally, joint training contributes with its regularizing effects, 
and drastic improvements are less 
expected\footnote{The source code and trained models are 
publicly released at \url{https://github.com/tianran/glimvec}, 
where we also show the mean performance and deviations of 
multiple random initializations, to give a more 
complete picture.}.
When compositional training is enabled, the system usually 
achieves better MR, though not always improves in other 
measures. The performance gains are more obvious on the 
WN18RR and FB15k-237 datasets, possibly because WN18 and 
FB15k contain a lot of easy instances that can be solved 
by a simple rule \cite{dettmers2018conve}.

%to performance by its 
%regularizing effects, 
%the joint training technique works more or less like a regularizer, 
%so we do not expect drastic gains. % in performance. 

Furthermore, the numbers demonstrated by our joint and base models are among 
the strongest in the literature. We have conducted re-experiments of several representative 
algorithms, and also compare with state-of-the-art published results. 
For re-experiments, we use \citet{DBLP:conf/aaai/LinLSLZ15}'s 
implementation\footnote{\url{https://github.com/thunlp/KB2E}} of 
TransE \citep{DBLP:conf/nips/BordesUGWY13} and TransR, 
which represent relations as vector translations; and \citet{DBLP:conf/aaai/NickelRP16}'s 
implementation\footnote{\url{https://github.com/mnick/holographic-embeddings}} of 
RESCAL \citep{Nickel:2011:TMC:3104482.3104584} and HolE, where RESCAL is most similar to 
the \textsc{base} model and HolE is a more parameter-efficient variant. We experimented with the default settings, 
and found that our models outperform most of them. 

Among the published results, STransE \citep{nguyen-EtAl:2016:N16-1} and ITransF \citep{xie-EtAl:2017:Long} are more complicated versions of TransR, achieving the previous highest MR on WN18 
but are outperformed by our \textsc{joint+comp} model. ITransF is most similar to 
our \textsc{joint} model in that they both learn sparse codings for relations. 
On WN18RR and FB15k-237, \citet{dettmers2018conve}'s report of 
ComplEx \citep{DBLP:conf/icml/TrouillonWRGB16} and ConvE were previously the best results. 
Our models mostly outperform them. 
Other results 
include \citet{kadlec-bajgar-kleindienst:2017:RepL4NLP}'s simple but strong baseline and several recent 
models 
\citep{DBLP:journals/corr/SchlichtkrullKB17,DBLP:conf/aaai/ShiW17,shen-EtAl:2017:RepL4NLP1} which achieve best results on FB15k or WN18 in some measure. Our models have comparable results. 






% ComplEx \citep{DBLP:conf/icml/TrouillonWRGB16} uses vectors with complex values.
% \citet{dettmers2018conve} による ComplEx の再実装は WN18RR における MR と H10 が最も高い.
% 同じく \citet{dettmers2018conve} が提案した ConvE は FB15k-237 でこれまでの SOTA を大幅に更新したが,我々はそれをさらに更新.

% However both of them achieved almost the same best MR on WN18 among previous methods,
% our \textsc{joint+comp} and \textsc{base+comp} model were higher.


% Our 
% re-experiments include: TransE \citep{DBLP:conf/nips/BordesUGWY13}, 



% Firstly, we compare with representative KBC models.

% %todo: related work

% TransE \citep{DBLP:conf/nips/BordesUGWY13} represents each relation $r$
% by a translation vector $\vec{v}_r$,
% which is chosen so that $\vec{v}_h + \vec{v}_r \approx \vec{v}_t$.
% RESCAL \citep{Nickel:2011:TMC:3104482.3104584}, a.k.a. the bilinear model,
% is most similar to our \textsc{base} model,
% which do not distinguish the vector for head entities from the vector for tail entities.

% To compare on WN18RR and FB15k-237, we re-experimented previous methods.
% For TransE and TransR \citep{DBLP:conf/aaai/LinLSLZ15},
% which introduces relation-specific spaces to TransE,
% we use an implementation provided by
% \citet{DBLP:conf/aaai/LinLSLZ15}\footnote{\url{https://github.com/thunlp/KB2E}}.
% For RESCAL and HolE \citep{DBLP:conf/aaai/NickelRP16},
% which uses a circular correlation instead of a tensor product,
% we use an implementation provided by
% \citet{DBLP:conf/aaai/NickelRP16}\footnote{\url{https://github.com/mnick/holographic-embeddings}}.
% Results are shown in the third sector of Table~\ref{tab:main-results}.
% TransR generally improves performance on all datasets especially in FB15k-237 compared with TransE.
% RESCAL suffers from a massive parameter space, which leads to its relatively low performance.
% HolE の結果は hogehoge...
% Note that our methods have consistently higher performance.

% We compare with previous state-of-the-art models
% from the literatures (the fourth sector of Table~\ref{tab:main-results}).

% STransE \citep{nguyen-EtAl:2016:N16-1} and ITransF \citep{xie-EtAl:2017:Long} are
% more complicated version of TransR.
% However both of them achieved almost the same best MR on WN18 among previous methods,
% our \textsc{joint+comp} and \textsc{base+comp} model were higher.

% ComplEx \citep{DBLP:conf/icml/TrouillonWRGB16} uses vectors with complex values.
% \citet{dettmers2018conve} による ComplEx の再実装は WN18RR における MR と H10 が最も高い.
% 同じく \citet{dettmers2018conve} が提案した ConvE は FB15k-237 でこれまでの SOTA を大幅に更新したが,我々はそれをさらに更新.

% Also,
% for WN18,
% R-GCN+~\citep{DBLP:journals/corr/SchlichtkrullKB17} has the best H10,
% but our \textsc{joint} and \textsc{base} are comparable.
% For FB15k,
% ProjE wlistwise \citep{DBLP:conf/aaai/ShiW17} and IRN \citep{shen-EtAl:2017:RepL4NLP1}
% show the best MR and H10.

%TransE~\citep{DBLP:conf/nips/BordesUGWY13},
%TransH~\citep{DBLP:conf/aaai/WangZFC14},
%TransR~\citep{DBLP:conf/aaai/LinLSLZ15},
%STransE~\citep{DBLP:conf/naacl/NguyenSQJ16},
%ITransF~\citep{xie-EtAl:2017:Long},
%HolE~\citep{DBLP:conf/aaai/NickelRP16},
%RESCAL~\citep{Nickel:2011:TMC:3104482.3104584},
%DistMult~\citep{Yang2015},
%ComplEx~\citep{DBLP:conf/icml/TrouillonWRGB16},
%ConvE~\citep{dettmers2018conve},
%Inverse Model~\citep{dettmers2018conve},
%IRN~\citep{shen-EtAl:2017:RepL4NLP1},
%Single DistMult~\citep{kadlec-bajgar-kleindienst:2017:RepL4NLP},
%Ensemble DistMult~\citep{kadlec-bajgar-kleindienst:2017:RepL4NLP},
%\texttt{fastText} - train~\citep{armand2017arxiv},
%\texttt{fastText} - train+valid~\citep{armand2017arxiv},
%R-GCN+~\citep{DBLP:journals/corr/SchlichtkrullKB17}.
%ProjE wlistwise~\citep{DBLP:conf/aaai/ShiW17}

%Our proposed method (the second block) achieves state-of-the-art performance
%for all metrics on FB15k-237,
%for some metrics on the other datasets.
%Strikingly, our method drastically improve MR on WN18, WN18RR, and FB15k-237.

%The third block shows ablation models.
%Without compositional training,
%we obtain further gain in all metrics on FB15k,
%Hits@10 on WN18, MRR and Hits@10 on WN18RR.
%It is not surprising for WN18 and WN18RR because
%compositional training はデータの背後に
%compositional constraint $\mat{M}_1 \cdot \mat{M}_2 \approx \mat{M}_3$
%を仮定しますが,直感的に,それらのデータセットにはこの constraint を満たすような関係は存在しないからです.
%%%% パス情報を使う先行研究の知見と合わない

%On the other hands, with autoencoder shows consistently better performance.
%%%% 本当?

%Let us compare these results with the results of previous studies.

\subsection{Intuition and Insight}\label{sec:analyzeautoenc}

What does the autoencoder look like? How does joint training affect 
relation matrices? We address these questions by analyses 
showing that \textbf{(i)} the autoencoder learns sparse and interpretable 
codings of relations, \textbf{(ii)} the joint training drives relation matrices 
toward a low dimension manifold, and \textbf{(iii)} it helps discovering compositional 
constraints.

\subsubsection*{Sparse Coding and Interpretability}\label{sec:interpretability}

Due to the $\relu$ function in \eqref{eq:coding}, our autoencoder learns 
sparse coding, with 
most relations having large code values at only two or three dimensions. 
This sparsity makes it easy to find patterns in the 
model that to some extent explain the semantics of relations.
Figure~\ref{fig:code-heatmap} shows 
some examples. 

In the first group of Figure~\ref{fig:code-heatmap}, we show a small number 
of relations that are almost always assigned a near one-hot coding, 
regardless of initialization. 
%Firstly, regardless of initialization, ... a fixed set of relations that ...
%These are shown as the first group of Figure...
These are high frequency relations joining two large categories 
(e.g. film and language), which probably constitute the skeleton of a KB. 

In the second group, we found the $12$th dimension strongly correlates with 
\texttt{currency}; and in the third group, we found the $4$th dimension 
strongly correlates with \texttt{film}. As for the relation 
\texttt{currency\_of\_film\_budget}, it has large code values at both dimensions. 
This kind of relation clustering also seems independent of initialization. 
Intuitively, it shows that the autoencoder may discover similarities 
between relations and 
promote indirect parameter sharing among them. 
Yet, as the autoencoder only reconstructs \emph{approximations} of relation 
matrices but never constrain them to be exactly equal to the original, relation 
matrices with very similar codings may still differ considerably. For 
example, 
\texttt{producer\_of\_film} and \texttt{writer\_of\_film} have
codings of cosine similarity 
0.973, but their relation matrices only 
have\footnote{Cosine similarity 0.338 is still high for matrices, due to the high 
dimensionality of their parameter space.} a 
cosine similarity 0.338. 


% We provide some intuitive examples to show how parameter is shared between different relation matrices.
% As we mentioned in Section~\ref{sec:jointtraining},
% because $\mat{A}, \mat{B}$ are shared by all relations,
% the coding $\vec{c}_r$ captures the association between relations in low dimensions.
% Hence, we visualize the coding $\vec{c}_r$ for some relations on FB15k-237 (Figure~\ref{fig:code-heatmap}).


% Although we do not impose any explicit hard constraints,
% we can see the codings obtain sparsity.
% Actually, most codings are expressed as combinations of only two or three
% representative dimensions (or \emph{prototypes}).
% This sparsity provides greater interpretability to our model.

% Firstly,
% the first group of codings
% use a dimension that other relations rarely use
% as the most important signal
% (e.g., the sixth dimension for \texttt{profession},
% the sixteenth dimension for \texttt{profession}$^{-1}$, and
% the first dimension for \texttt{film\_crew\_role}$^{-1}$).
% A common characteristic among these relations is a high degree,
% that is,
% they have a high frequency in the training set and
% their relation categories are not \onetoone.
% Since such relations need many parameters to learn,
% our model seems to assign individual dimensions to them.
% Secondly,
% the second group contains codings related to the \emph{currency} relation.
% These relations use the 12th dimension as the most important signal; thus
% we can say that the 12th dimension represents the prototype of \emph{currency}.
% Thirdly,
% the third group contains codings related to the \emph{film} relation.
% These relations use the 4th dimension as the most important signal; thus
% we can say that the the 4th dimension represents the prototype of \emph{film}.
% It is worth noting that
% the coding of \texttt{currency\_of\_film\_budget} related to both the \emph{currency} and the \emph{film} relation
% use both the 4th and the 12th dimensions as important signals,
% our joint training captures surprisingly well the association between relations.

% Because the associated relations represent different transformations strictly,
% our model should learn different relation matrices,
% while its codings are very similar.
% Actually,
% \texttt{currency\_of\_country} and \texttt{currency\_of\_company}
% has very similar codings and the cosine similarity was 0.796,
% but the cosine similarity between these relation matrices was 0.281.
% Conversely,
% cosine similarities between \texttt{producer\_of\_film} and \texttt{writer\_of\_film}
% were 0.973 and 0.338, respectively.

% Note that these observations were valid across different initialization values (APPENDIX).


\subsubsection*{Low dimension manifold}


\begin{figure}[!t]
\centering
\begin{subfigure}[b]{0.48\columnwidth}
\centering
\includegraphics[width=\textwidth]{base}
% http://www.cl.ecei.tohoku.ac.jp/~ryo-t/dcsveckb/model-nobr-noautoenc-nocomp-2048_umap.html
\caption{\textsc{base}}
\label{subfig:umap-base}
\end{subfigure}
\begin{subfigure}[b]{0.48\columnwidth}
\centering
\includegraphics[width=\textwidth]{joint}
% http://www.cl.ecei.tohoku.ac.jp/~ryo-t/dcsveckb/model-nobr-nocomp-2048_umap.html
\caption{\textsc{joint}}
\label{subfig:umap-joint}
\end{subfigure}
\begin{subfigure}[b]{0.48\columnwidth}
\centering
\includegraphics[width=\textwidth]{base+comp}
% http://www.cl.ecei.tohoku.ac.jp/~ryo-t/dcsveckb/model-nobr-noautoenc-nolex-1.0-2048_umap.html
\caption{\textsc{base+comp}}
\label{subfig:umap-base+comp}
\end{subfigure}
\begin{subfigure}[b]{0.48\columnwidth}
\centering
\includegraphics[width=\textwidth]{joint+comp}
% http://www.cl.ecei.tohoku.ac.jp/~ryo-t/dcsveckb/model-nobr-nolex-1.0-2048_umap.html
\caption{\textsc{joint+comp}}
\label{subfig:umap-joint+comp}
\end{subfigure}
\caption{%
By UMAP, relation matrices are embedded into a 2D plane. 
%Visualization of relation matrices by UMAP. Each dot is a high dimension matrix 
%projected to the 2D plane. 
Colors show frequencies of relations; and lighter color means more frequent.}
%Comparison of UMAP embeddings for relations matrices between \textsc{base+comp} and the \textsc{joint+comp}.
%The darker the color, the lower the frequency of the relation in the training set.}
\label{fig:umap}
\end{figure}

In order to visualize the relation matrices learned by our joint and base 
models, we use 
UMAP\footnote{\url{https://github.com/lmcinnes/umap}} 
\citep{2018arXivUMAP} to 
embed $\mat{M}_r$ into a 2D 
plane\footnote{UMAP is a recently proposed manifold 
learning algorithm based on the fuzzy topological structure. 
We also tried 
t-SNE \citep{maaten2008visualizing} but found UMAP more insightful.}. 
We use relation matrices trained on FB15k-237, and compare models trained by 
the same number of epochs. 
The results are shown in Figure~\ref{fig:umap}. 

We can see that Figure~\ref{subfig:umap-base} and Figure~\ref{subfig:umap-base+comp} are mostly similar, with high frequency 
relations scattered randomly around a low frequency cluster, suggesting that 
they come from various directions of a high dimension space, with 
frequent relations probably being pulled further by the training updates. 
On the other hand, in Figure~\ref{subfig:umap-joint} and Figure~\ref{subfig:umap-joint+comp} we found less frequent relations being 
clustered with frequent ones, and multiple traces of low dimension 
structures. It suggests that joint training with an autoencoder 
indeed drives relations toward a low dimension manifold. In addition, 
Figure~\ref{subfig:umap-joint+comp} shows different structures against Figure~\ref{subfig:umap-joint}, which we conjecture could be 
related to compositional constraints discovered by compositional training. 


% To show that joint training drives relations toward a low-dimensional manifold,
% in Figure~\ref{fig:umap}
% we plot relation matrices in the 2D space using UMAP \citep{2018arXivUMAP},
% which is a non-linear dimension reduction technique similar to t-SNE \citep{maaten2008visualizing}.

% In \textsc{base+comp} (Figure~\ref{subfig:noautoenc}),
% lower frequency relations form a large cluster in the center of Figure~\ref{subfig:noautoenc}
% and higher frequency relations are spread out radially around the cluster.
% In \textsc{joint+comp} (Figure~\ref{subfig:full-model}),
% lower frequency relations still form a large cluster,
% but some parts of the cluster spreads away from the center,
% and whole relations are arranged into 潰れた感じに.
% This suggests that ...

% Note that these observations were valid across different initialization values and UMAP parameters (APPENDIX).

\subsubsection*{Compositional constraints}\label{sec:compositionalconstraints}

In order to directly evaluate a model's ability to find compositional constraints, 
we extracted from FB15k-237 a list of $(r_1/r_2, r_3)$ pairs such that 
$r_1/r_2$ matches $r_3$. Formally, the list is constructed as below. 
%built a dataset from FB15k-237 as follows. 
%if joint training with an autoencoder helps discovering compositional constraints, 
For any relation $r$, we define 
a \emph{content set} $C(r)$ as the set of $(h,t)$ pairs such that 
$\langle h,r,t\rangle$ is a fact in the KB. Similarly, we define 
$C(r_1/r_2)$ as the set of $(h,t)$ pairs such that 
$\langle h,r_1/r_2,t\rangle$ is a path.
We regard $(r_1/r_2, r_3)$ as a 
compositional constraint if their content sets are similar; 
that is, 
if $\lvert C(r_1/r_2)\cap C(r_3) \rvert\geq 50$ and 
the Jaccard similarity between $C(r_1/r_2)$ and $C(r_3)$ is $\geq 0.4$. Then, after 
filtering out degenerated cases such as $r_1=r_3$ or $r_2=r_1^{-1}$, we 
obtained a list of 154 compositional constraints, e.g. \\
(\texttt{currency\_of\_country}/\texttt{country\_of\_film}, 
\texttt{currency\_of\_film\_budget}). 

\begin{table}[!t]
\centering
\small
\setlength{\tabcolsep}{14pt}
\begin{tabular}{lrr}
\toprule
Model & MR & MRR \\
\midrule
\textsc{joint+comp} & \textbf{130$\pm$27} & \textbf{.0481$\pm$.0090} \\ 
\textsc{base+comp} & 150$\pm$3 & .0280$\pm$.0010 \\ 
\textsc{RandomM2} & 181$\pm$19 & .0356$\pm$.0100 \\
\bottomrule
\end{tabular}
\caption{Performance at discovering compositional constraints 
extracted from FB15k-237}
\label{tab:compositional-constraints}
\end{table}

For each compositional constraint $(r_1/r_2, r_3)$ in the list, 
we take the matrices $\mat{M}_1$, $\mat{M}_2$ and $\mat{M}_3$ 
corresponding to $r_1$, $r_2$ and $r_3$ respectively, and 
rank $\mat{M}_3$ according to its cosine similarity with $\mat{M}_1\mat{M}_2$, among all relation 
matrices. Then, we calculate MR and MRR for evaluation. 
We compare the \textsc{joint+comp} model 
to \textsc{base+comp}, as well as a randomized baseline where 
$M_2$ is selected randomly from the relation matrices in \textsc{joint+comp} instead (\textsc{RandomM2}).
The results are shown in Table~\ref{tab:compositional-constraints}. 
We have evaluated 5 different random initializations for each model, trained 
by the same 
number of epochs, and we report the mean and standard deviation. We verify 
that \textsc{joint+comp} performs better than \textsc{base+comp}, 
indicating that 
joint training with an autoencoder indeed helps discovering 
compositional constraints. Furthermore, the random baseline 
\textsc{RandomM2} tests a hypothesis that joint training might be just clustering $M_3$ and $M_1$ here, to the extent that 
$M_3$ and $M_1$ are so close that even 
a random $M_2$ can give the correct answer; but as it turns out, \textsc{joint+comp} largely 
outperforms \textsc{RandomM2}, excluding this possibility. Thus, 
joint training performs better not simply because 
it clusters relation matrices; it learns compositions indeed.

\subsection{Losses and Gains}

In the KBC task, where are the losses and what are the gains of 
different settings?
With additional evaluations, we show \textbf{(i)} some crucial 
settings for the base model, and 
\textbf{(ii)} joint training with an autoencoder benefits more from
compositional training. 
%joint training gains more as it focuses more on composition. %, 
%and \textbf{(iii)} the losses of compositional training. 

\subsubsection*{Crucial settings for the base model}\label{sec:crucialsettings}

\begin{table}[!t]
\centering
\small
\setlength{\tabcolsep}{8pt}
\begin{tabular}{rrrr}
\toprule
\multicolumn{1}{c}{Settings} & MR & MRR & H10 \\
\midrule
\multicolumn{1}{l}{\textsc{base}} & \textbf{214} & \textbf{.338} & \textbf{52.5} \\ % model-nobr-noautoenc-nocomp-512
\midrule
no normalization & 309 & .326 & 49.9 \\ % model-nonorm-512
no regularizer & 400 & .328 & 51.3 \\ % model-weakreg-512
pure Gaussian & 221 & .336 & 52.1 \\ % model-gaussinit-512
\qquad unigram distribution& 215 & .324 & 50.6 \\ % model-unigramnegsamp-512
\bottomrule
\end{tabular}
\caption{Ablation of the four settings of the base model as described 
in Sec.\ref{sec:trainingbase}}
\label{tab:crucial-settings}
\end{table}

It is noteworthy that our base model already achieves strong results. 
This is due to several detailed but crucial settings 
as we discussed in Sec.\ref{sec:trainingbase}; 
Table~\ref{tab:crucial-settings}
shows their gains on 
the FB15k-237 validation data. 
The most dramatic improvement comes from the regularizer that drives matrices 
to orthogonal. 

% To reveal crucial settings leading to our state-of-the-art results,
% We examine how performance varies given different initialization on FB15k-237 (Table~\ref{tab:crucial-settings}).
% \begin{table}[!t]
% \centering
% \setlength{\tabcolsep}{3pt}
% \begin{tabular}{@{}lccccc@{}}
% \toprule
% \multirow{2}{*}{Method} & & & \multicolumn{3}{c}{Hits} \\
% \cmidrule(l){4-6}
% & MR & MRR & @10 & @3 & @1 \\
% \midrule
% Proposed & 194 & .334 & 51.9 & 36.7 & 24.2 \\ % model-nobr-nolex-1.0-512
% init2 & 192 & .335 & 51.8 & 36.7 & 24.4 \\ % init2/model-nobr-nolex-1.0-512
% init3 & 197 & .334 & 51.9 & 36.8 & 24.2 \\ % init3/model-nobr-nolex-1.0-512
% init4 & 195 & .335 & 52.1 & 36.7 & 24.3 \\ % init4/model-nobr-nolex-1.0-512
% gaussinit & 221 & .336 & 52.1 & 36.7 & 24.5 \\ % model-gaussinit-512
% unigramnegsamp & 215 & .324 & 50.6 & 35.2 & 23.4 \\ % model-unigramnegsamp-512
% weakreg & 400 & .328 & 51.3 & 36.3 & 23.5 \\ % model-weakreg-512
% nonorm & 309 & .326 & 49.9 & 35.8 & 23.8 \\  % model-nonorm-512
% \bottomrule
% \end{tabular}
% \caption{Validation set, FB15k-237, same epochs}
% \label{tab:crucial-settings}
% \end{table}

\subsubsection*{Gains with compositional training}\label{sec:gainscomptrain}

One can force a model to focus more on (longer) compositions of relations, 
by sampling longer paths in compositional training. 
Since joint training with an autoencoder helps discovering compositional 
constraints, we expect it to be more helpful when the sampled paths are longer. 
In this work, path lengths are sampled from a Poisson distribution, we thus 
vary the mean $\lambda$ of the Poisson to control the strength of compositional 
training. The results on FB15k-237 are shown in Table~\ref{tab:ablation-ae-comp}. 

We can see that, as $\lambda$ gets larger, MR improves much but MRR slightly drops. 
It suggests that in FB15k-237, composition of relations might mainly help 
finding more appropriate candidates for a missing entity, rather than 
pinpointing a correct one. Yet, joint training improves base models even more 
as the paths get longer, especially in MR. 
It further supports our conjecture that joint training with an autoencoder 
may strongly interact with compositional training. 

%\section{Losses of compositional training}

%However, in our experiments ...


\begin{table}[!t]
\centering
\setlength{\tabcolsep}{5pt}
\small
\begin{tabular}{@{}lcrrrrrr@{}}
\toprule
\multirow{2}{*}{Model} & \multirow{2}{*}{$\lambda$} & \multicolumn{3}{c}{Valid} & \multicolumn{3}{c}{Test} \\
\cmidrule(lr){3-5} \cmidrule(l){6-8}
& & MR & MRR & H10 & MR & MRR & H10 \\
\midrule
\textsc{base}  & 0 & 209 & .341 & 52.9 & 215 & .337 & 52.3 \\ % model-nobr-noautoenc-nocomp-3584
\textsc{joint} & 0 & +1 & -.001 & -.2 & \textbf{-3} & -.001 & 0 \\ % model-nobr-nocomp-2048
\midrule
\textsc{base}  & 0.5 & 204 & .337 & 52.2 & 211 & .332 & 51.7 \\ % model-nobr-noautoenc-nolex-0.5-1024
\textsc{joint} & 0.5 & \textbf{-3} & \textbf{+.002} & \textbf{+.1} & +1 & \textbf{+.002} & \textbf{+.2} \\ % model-nobr-nolex-0.5-512
\midrule
\textsc{base}  & 1.0 & 191 & .334 & 52.0 & 203 & .328 & 51.5 \\ % model-nobr-noautoenc-nolex-1.0-1536
\textsc{joint} & 1.0 & \textbf{-5} & \textbf{+.002} & -.1 & \textbf{-6} & \textbf{+.003} & \textbf{+.1} \\ % model-nobr-nolex-1.0-2560
\bottomrule
\end{tabular}
\caption{%
Evaluation of \textsc{base} and gains by \textsc{joint}, on FB15k-237 
with different strengths of compositional training. Bold numbers are improvements.}
\label{tab:ablation-ae-comp}
\end{table}

\section{Conclusion}

We have investigated a dimension reduction technique which trains a KB embedding model jointly with an autoencoder. 
%In this paper, we have trained embedding models for knowledge bases (KBs) and 
%investigated a technique which reduces the dimensionality of relation parameters 
%by jointly training with an autoencoder. 
We have developed new training techniques and achieved 
state-of-the-art results on 
several KBC tasks with strong improvements in Mean Rank. 
Furthermore, we have shown that the autoencoder learns low dimension sparse 
codings that can be easily explained; the joint training technique 
drives high-dimensional data toward low dimension manifolds; and the 
reduction of dimensionality may interact strongly 
with composition, help discovering compositional constraints 
and benefit from compositional training. 
We believe these findings provide insightful understandings of KB embedding models and might be applied to other neural networks beyond the KBC task.


% semantic parsing~\citep{berant-EtAl:2013:EMNLP},
% question answering~\citep{hixon-clark-hajishirzi:2015:NAACL-HLT},
% information extraction~\citep{DBLP:journals/pieee/Nickel0TG16},


% Embedding of graph structures is also universally addressed in NLP, 
% for both text and linked data (cite..),
% we ourselves borrow from an embedding model for dependency trees \citep{tian-okazaki-inui:2016:P16-1}, 
% and training with external textual knowledge \citep{} would be a promising 
% future direction. 
% \citep{riedel-EtAl:2013:NAACL-HLT}




%Recently, various embedding models have been proposed for KBC.
%For further details of such models,
%refer to the surveys by \citet{DBLP:journals/corr/Nguyen17a} and \citet{DBLP:journals/tkde/WangMWG17}.

%In the seminal work by \citet{DBLP:conf/nips/BordesUGWY13}, they proposed TransE
%which regards the relation as a translation vector $\vec{v}_r$ between two entity vectors $\vec{v}_h$ and $\vec{v}_t$.
% which represents relationships as translations in the embedding space.
%TransE is suitable for 1-to-1 relations, but has flaws for 1-to-N, N-to-1, and N-to-N relations.
%To overcome the limitations, extensions such as
%TransR \citep{DBLP:conf/aaai/LinLSLZ15} and STransE \citep{DBLP:conf/naacl/NguyenSQJ16}
%set a projection matrix $\mat{M}_r$ for each relation $r$ to project entity embedding into relation vector space.
%However, These extensions introduce too many additional parameters
%and lead to the difficulty of parameter sharing among relations.

%Given this background, \citet{xie-EtAl:2017:Long} have same motivation with us and
%model parameter sharing among relations explicitly.
%Their model ITransF is based on STransE.
%They added a \emph{hard} constraint that a linear projection for each relation must be represented
%in the sum of a few common concept matrices,
%and employed sparse attention to learn KB embeddings with this constraint.
%In contrast to ITransF, since our model do not have such \emph{hard} constraints,
%parameters can be optimized with stochastic gradient descent (SGD).
%Hence, our model can be seen as more flexible one.

% 知識ベース補完にはパス情報が有用であることも知られている~\cite{DBLP:conf/emnlp/GuML15,DBLP:conf/emnlp/LinLLSRL15}が,
% 我々の同時学習手法は訓練スキームに特別な制約を入れていないので,パス情報
% を利用する訓練を取り入れることも簡単である.パス情報を使った訓練は,多段階
% 推論によってより多くの欠損事実を補完する可能性を持つ.
% 例えば,バラク・オバマがホノルルで生まれ,かつホノルルがアメリカ合衆国の
% 都市であることを使えば,オバマの国籍がアメリカ合衆国であることを推測できる
% かもしれない.したがって,今回我々はパス情報も取り入れた訓練モデルを
% 実装し,評価した.将来的には更に,Riedelら(2013)~\cite{DBLP:conf/naacl/RiedelYMM13}の研究のように,自然言語のテキストも使った同時学習で知識ベース補完を行いたい.

% Some studies model the intermediate entities in the relation paths
% \citep{toutanova-EtAl:2016:P16-1,das-EtAl:2017:EACLlong1}.
% They do not deal with the data sparseness problem, and do not compare their models with other state-of-the-art models.

%An autoencoder is a neural network which is trained to reconstruct a given input from its latent representation.
%It has been used to pre-train networks in an unsupervised fashion \citep{Erhan:2010:WUP}
%or learn distributed representations \citep{socher-EtAl:2011:EMNLP,silberer-lapata:2014:P14-1,titov-khoddam:2015:NAACL-HLT}
%to make use of the ability of dimensionality reduction.
%However, to the best of our knowledge, no previous work uses the autoencoder as a means to
%promote parameter sharing by joint training.

\section*{Acknowledgments}

This work was supported by JST CREST Grant Number JPMJCR1301, Japan. We thank 
Pontus Stenetorp, Makoto Miwa, and the anonymous reviewers for many helpful 
advices and comments.

\bibliographystyle{acl_natbib}
\bibliography{acl2018}

\appendix

% \section{Losses of compositional training}

% To gain better insights into the improvements of our joint training,
% we evaluate MR and MRR results on FB15k-237 with respect to the relation categories.
% Following \citet{DBLP:conf/nips/BordesUGWY13}, we categorize the relations according to the cardinalities
% of their associated head and tail entities in four types: \onetoone, \oneton, \ntoone, and \nton.
% For each relation $r$, we calculate the averaged number $a_h$ of heads $h$ for a pair $(r, t)$
% and averaged number $a_t$ of tails $t$ for a pair $(h, r)$.
% If $a_h < 1.5$ and $a_t < 1.5$, then $r$ is categorized \onetoone.
% If $a_h < 1.5$ and $a_t \ge 1.5$, then $r$ is categorized \oneton.
% If $a_h \ge 1.5$ and $a_t < 1.5$, then $r$ is categorized \ntoone.
% If $a_h \ge 1.5$ and $a_t \ge 1.5$, then $r$ is categorized \nton.
% We obtain that FB15k-237 has
% 7.1\% of \onetoone relations,
% 11.0\% of \oneton,
% 36.3\% of \ntoone, and
% 45.6\% of \nton.

% Table~\ref{tab:category-wise} shows the results.
% \begin{table*}[!t]
% \centering
% \setlength{\tabcolsep}{3pt}
% \begin{tabular}{@{}llcccccccc@{}}
% \toprule
% \multirow{2}{*}{Model} & & \multicolumn{4}{c}{Predicting head $h$} & \multicolumn{4}{c}{Predicting tail $t$} \\
% \cmidrule(lr){3-6} \cmidrule(l){7-10}
% & & \onetoone & \oneton & \ntoone & \nton & \onetoone & \oneton & \ntoone & \nton \\
% \midrule
% \multirow{2}{*}{\textsc{base}}  & MR  & 254 & 1019 & 44 & 114 & 231 & 74 & 568 & 186 \\ % model-nobr-noautoenc-nocomp-3584
%                                 & MRR & .470 & \textbf{.097} & .761 & \textbf{.371} & \textbf{.466} & .448 & .091 & \textbf{.265} \\
% \multirow{2}{*}{\textsc{joint}} & MR  & 216 & 1072 & 48 & 116 & 195 & 70 & \textbf{563} & 182 \\ % model-nobr-nocomp-2048
%                                 & MRR & .456 & .086 & .760 & \textbf{.371} & .449 & .448 & \textbf{.092} & .263 \\
% \midrule
% \multirow{2}{*}{\textsc{base+comp}}  & MR  & 148 & \textbf{727} & 38 & 94 & \textbf{166} & \textbf{68} & 705 & \textbf{141} \\ % model-nobr-noautoenc-nolex-1.0-1536
%                                      & MRR & .471 & .071 & .761 & .361 & .463 & .450 & .080 & .261 \\
% \multirow{2}{*}{\textsc{joint+comp}} & MR  & \textbf{144} & 739 & \textbf{32} & \textbf{92} & 195 & 70 & 658 & 143 \\ % model-nobr-nolex-1.0-2560
%                                      & MRR & \textbf{.473} & .073 & \textbf{.763} & .365 & .453 & \textbf{.451} & .080 & .261 \\
% \bottomrule
% \end{tabular}
% \caption{%
% Detailed results by category of relations on the validation set of FB15k-237.}
% \label{tab:category-wise}
% \end{table*}
% \begin{outline}
% \1 comp すると \oneton prediction (\oneton relation while predicting head $h$ and \ntoone relation while predicting tail $t$) が難しくなる
% \1 
% \end{outline}

\section{Out-of-vocabulary Entities in KBC}

Occasionally, a KBC test set may contain entities that never appear in the training data. Such 
out-of-vocabulary (OOV) entities pose a challenge to KBC systems; while some systems address 
this issue by explicitly learn an OOV entity vector \citep{dettmers2018conve}, our approach is described below. For an incomplete triple $\mtriple{h}{r}{?}$ in the test, if $h$ is OOV, 
we replace it with the most frequent entity that has ever appeared as a head of relation $r$ 
in the training data. If the gold tail entity is OOV, we use the zero vector for computing 
the score and the rank of the gold entity. 

Usually, OOV entities are rare and negligible in evaluation; except for the WN18RR test data 
which contains about 6.7\% triples with OOV entities. Here, we also report adjusted scores on WN18RR 
in the setting that all triples with OOV entities are removed from the test. The 
results are shown in Table~\ref{tab:wn18rr-remove-oov}.

\begin{table}[ht]
\centering
\small
\setlength{\tabcolsep}{13pt}
\begin{tabular}{lccc}
\toprule
Model & MR & MRR & H10 \\
\midrule
\textsc{joint} & \textbf{3317} & \textbf{.493} & \textbf{57.2} \\
\textsc{base} & 3435 & .492 & 56.7 \\
\midrule
\textsc{joint+comp} & \textbf{1507} & \textbf{.367} & \textbf{58.7} \\
\textsc{base+comp} & 1629 & .332 & 58.0 \\
\bottomrule
\end{tabular}
\caption{Adjusted scores on WN18RR.}
\label{tab:wn18rr-remove-oov}
\end{table}

% there are entities 

% WN18RR contains 212 out-of-vocabulary (OOV) entities
% (i.e., they do not appear in the training set) in the test set.
% Since many test triples (about 6.7\% out of all the test triples) has an OOV entity in their head or tail
% and can affect the overall performance,
% we use the following procedure while testing such triples.
% When we predict $h$ given $\mtriple{?}{r}{t}$ s.t. $h$ is an OOV entity,
% we assume $\vec{v}_h$ is a zero vector and the rest of procedure is identical to the above.
% When we predict $h$ given $\mtriple{?}{r}{t}$ s.t. $t$ is an OOV entity,
% we replace
% $t$ with
% pseudo $t'$ that most frequently occurs
% in the training set in the form $\mtriple{?}{r}{t'}$,
% and the rest is identical to the above.
% FB15k-237 also has a small number of triples that contain OOV entities in the test set,
% but for simplicity we omit such triples from the test set.

% \section{Initialization}

\end{document}
